%%%%%%%%%%%%%%%%
%   Preamble   %
%%%%%%%%%%%%%%%%

%%%%%%%%%%%%%%%%%%%%%%
%   Begin preamble   %
%%%%%%%%%%%%%%%%%%%%%%

% !TEX TS-program = pdflatex
% !TEX encoding = UTF-8 Unicode

% Handout versions
% \documentclass[handout]{beamer}
% \usepackage{pgfpages}
% \pgfpagesuselayout{2 on 1}[a4paper, border shrink = 5mm]
% \pgfpagesuselayout{4 on 1}[a4paper, border shrink = 5mm, landscape]

% Slide version
\documentclass[aspectratio = 169, 9pt]{beamer}

% Margins
\setbeamersize{text margin left = .3cm, text margin right = .3cm}
\geometry{hmargin = 2.56cm, vmargin = 0cm}

% Suppresses navigation symbols
\setbeamertemplate{navigation symbols}{}

% Colors
\definecolor{main-color}{RGB}{0,51,102} % dark blue

\definecolor{gray}{gray}{0.50}
\definecolor{dark-gray}{gray}{0.40}
\definecolor{light-gray}{gray}{0.95}

\definecolor{blue}{RGB}{0,114,178}
\definecolor{dark-blue}{RGB}{0,51,102} % dark blue
\definecolor{light-blue}{RGB}{86,180,233}

\definecolor{red}{RGB}{213,94,0}
\definecolor{dark-red}{RGB}{139,0,0}

\definecolor{green}{RGB}{0,158,115}
\definecolor{dark-green}{RGB}{0,100,80}

\definecolor{purple}{RGB}{204,121,167}
\definecolor{dark-purple}{RGB}{90,0,90}

\definecolor{yellow}{RGB}{240,228,66}
\definecolor{light-yellow}{RGB}{255,239,158}
\definecolor{orange}{RGB}{230,159,0}
\definecolor{brown}{RGB}{166,86,40}

% Beamer Colors
\setbeamercolor{structure}{fg = main-color, bg = white}
\setbeamercolor{block title}{fg = black, bg = white}
\setbeamercolor{block body}{fg = black, bg = white}

\setbeamercolor{title in head/foot}{fg = main-color, bg = white}

\setbeamercolor{titlepage headline}{fg = white, bg = white}
\setbeamercolor{titlepage footline}{fg = white, bg = white}

\setbeamercolor{frame headline}{fg = white, bg = white}
\setbeamercolor{frame footline}{fg = main-color, bg = white}

% Headline
\defbeamertemplate*{headline}{myheadline}
 {
   \ifnum \insertpagenumber = 1
     {
     \begin{beamercolorbox}[wd = \paperwidth, ht = 5pt]{titlepage headline}%
     \end{beamercolorbox}%
     }
     \else
     {
     \begin{beamercolorbox}[wd = \paperwidth, ht = 5pt]{frame headline}%
     \end{beamercolorbox}%
     }
     \fi
 }

% Footline (with numbers)
 \defbeamertemplate*{footline}{myfootline}
 {
   \ifnum \insertpagenumber = 1
     {
     \begin{beamercolorbox}[wd = \paperwidth, ht = 15pt]{titlepage footline}%
     \end{beamercolorbox}%
     }
     \else
     {
     \begin{beamercolorbox}[wd = \paperwidth, ht = 15pt, right, rightskip = 5pt]{frame footline}%
       \vbox to15pt{\vfil\hbox{\insertframenumber{}}\vfil}%
     \end{beamercolorbox}%
     }
     \fi
 }

% Beamer-specific font parameters
\usefonttheme{structurebold}

% Frame title
\setbeamerfont{frametitle}{size = \LARGE}
\setbeamerfont{framesubtitle}{size = \normalsize}

% General typography
\usepackage{roboto}
\usepackage[T1]{fontenc}
\usepackage[utf8]{inputenc}
\usepackage[protrusion = true, expansion = true]{microtype}
\usepackage{csquotes}
\usepackage[english]{babel}
\clubpenalty = 10000
\widowpenalty = 10000

% Spacing
\linespread{1.25}                     % Space between lines
\setlength{\parskip}{\bigskipamount}  % Space between paragraphs
\setlength{\parindent}{0pt}           % Indentation
\setlength{\leftmargini}{0.45cm}      % Left margin of items
\setbeamertemplate{itemize/enumerate body begin}{\vspace{-2ex}}

% Spacing for equations
\AtBeginDocument{%
 \abovedisplayskip = -15pt plus 3pt minus 3pt
 \abovedisplayshortskip = -15pt plus 3pt minus 3pt
 \belowdisplayskip = 5pt plus 3pt minus 3pt
 \belowdisplayshortskip = 5pt plus 3pt minus 3pt
}

% Math
\usepackage{amsthm}
\usepackage{amssymb}
\usepackage{bbm}
\usepackage{bm}
\usepackage{mathtools}

% Footnotes
\usepackage[flushmargin, hang]{footmisc}
\setlength{\footnotemargin}{0.7em}
\setlength{\footnotesep}{1\baselineskip}

% Improve exhibit captions
\usepackage[font = footnotesize, labelfont = bf, textfont = sl, labelsep = colon, justification = centering, labelformat = empty]{caption}

% Graphs
\usepackage{graphicx}
\usepackage{wrapfig}                  % To add lateral floats

% Tables
\usepackage{booktabs}                 % To use toprule, midrule and bottomrule
\usepackage{longtable}                % To span multiple pages
\usepackage{array}
\usepackage{multirow}
\usepackage{threeparttable}
\usepackage{threeparttablex}          % Add notes under a longtable
\usepackage{subcaption}
\usepackage{makecell}
\usepackage{float}
\usepackage{dcolumn}

% Links
\hypersetup{colorlinks = true, citecolor = blue, urlcolor = blue, linkcolor = blue}

% Highlight
\usepackage{soul}
\sethlcolor{light-yellow}

\makeatletter
\let\HL\hl
\renewcommand\hl{%
  \let\set@color\beamerorig@set@color
  \let\reset@color\beamerorig@reset@color
  \HL}
\makeatother

% Show overfull cases
\overfullrule = 2cm

% Appendix
\usepackage{appendixnumberbeamer}

%%%%%%%%%%%%%%%%%%%%%%%%%%%%%%%%%%%%%
%   tikz: call-outs, diagrams etc   %
%%%%%%%%%%%%%%%%%%%%%%%%%%%%%%%%%%%%%

\usepackage{tikz}
\usetikzlibrary{arrows, shapes, trees}
\tikzset{every picture/.style = {remember picture}}

%%%%%%%%%%%%%%%%%%%%
%   New commands   %
%%%%%%%%%%%%%%%%%%%%

% Numerical constants
\providecommand*{\eu}{\ensuremath{\mathrm{e}}} % Euler

% Operators
\providecommand*{\dnorm}{\phi} % normal density
\providecommand*{\pnorm}{\Phi} % normal distribution function

\providecommand*{\diff}{\mathop{}\!\mathrm{d}} % differential
\providecommand*{\loglik}{\ell} % log likelihood

\providecommand*{\medtimes}{\mathbin{\scalebox{1.1}{\ensuremath{\times}}}} % medtimes
\providecommand*{\medcup}{\mathbin{\scalebox{1.5}{\ensuremath{\cup}}}} % medcup
\providecommand*{\medcap}{\mathbin{\scalebox{1.5}{\ensuremath{\cap}}}} % medcap

% Modifiers
\providecommand*{\upperbar}[1]{\mkern 1.5mu\overline{\mkern-1.5mu#1\mkern-0.1mu}\mkern 1.5mu} % upperbar
\providecommand*{\lowerbar}[1]{\mkern 1.5mu\underline{\mkern-1.5mu#1\mkern-0.1mu}\mkern 1.5mu} % lowerbar

% Other formatting rules
\providecommand*{\mat}[1]{\mathbf{#1}} % matrix names
\providecommand*{\vec}[1]{\boldsymbol{#1}} % vector names 

% Make hfill compatible with Beamer
\newcommand{\myhfill}{\hskip0pt plus 1filll}

%%%%%%%%%%%%%%%%%%%%
%   End preamble   %
%%%%%%%%%%%%%%%%%%%%


%%%%%%%%%%%%%%%%%%%%
%   Presentation   %
%%%%%%%%%%%%%%%%%%%%

\begin{document}

\begin{frame}{}

\centering

\bigskip \bigskip {\huge \textcolor{main-color}{\textbf{When You Can't Afford to Wait for a Job:}}} \\ \medskip {\huge \textcolor{main-color}{\textbf{The Role of Time Discounting for Own-Account}}} \\ \medskip {\huge \textcolor{main-color}{\textbf{Workers in Developing Countries}}}

\bigskip

\begin{tabular}{ccc}
\textbf{Thiago Scarelli}  & \quad & \textbf{David N. Margolis} \\
Paris School of Economics & \quad & Paris School of Economics
\end{tabular}

\bigskip \bigskip \bigskip 3 September 2020

\makebox[\linewidth]{\centering 3rd Annual IZA/World Bank/NJD/UNU-WIDER Jobs and Development Conference}

\end{frame} %%%%%%%%%%%%%%%%%%%%%%%%%%%%%%%%%%%%%%%%%%%%%%%%%%%%%%%%%%%%%%%%%%%%%

\begin{frame}{How much own-account work (OAW) is constrained?}

\pause \textbf{Motivation:}

\begin{itemize}
  \item In developing countries, 27\% of urban employed population are OAWs.
  \item Complex category: some are true entrepreneurs, some are constrained.
\end{itemize} \pause

\textbf{Questions:}

\begin{enumerate}
  \item Why do people take up OAW when mean income is below employees'?
  \item Why is OAW much more prevalent in poor regions?
  \item Among OAWs, how many are constrained?
\end{enumerate} \pause

\textbf{Our proposition:} 
\begin{itemize}
  \item Explore the time trade-off between \hl{OAW now} vs. \hl{better paid job later}.
\end{itemize} \pause

\fbox{\parbox{.95\textwidth}{\centering \emph{People may choose OAW because they have urgent consumption needs and can't afford to wait for a good job sometime in the future.}}}

\end{frame} %%%%%%%%%%%%%%%%%%%%%%%%%%%%%%%%%%%%%%%%%%%%%%%%%%%%%%%%%%%%%%%%%%%%%

\begin{frame}{}

\bigskip

\textbf{\textcolor{main-color}{\fbox{Step 1}} Incorporate OAW in a very simple job search framework:}

\begin{equation*}
\text{\hl{\textit{OAW is chosen if}}} \quad y > b + \frac{\lambda}{\delta + \rho} \cdot \int_{w_r}^{\infty} \Big( w - w_r \Big) \, dF(w)
\end{equation*}

\pause
\begin{itemize}
  \item OAW is more frequent where present value of looking for a job is lower.
  \item Low-pay OAW can be optimal if jobs are scarce and consumption is urgent.
\end{itemize} \medskip \pause

\textbf{\textcolor{main-color}{\fbox{Step 2}} Estimate the labor market parameters using survey data for Brazil:}

\begin{itemize}
  \item If I were to look for a job, how much could I expect to earn?
  \item For how long would I need to search? How long would such job last?
\end{itemize} \medskip \pause

\textbf{\textcolor{main-color}{\fbox{Step 3}} Infer the subjective time discount from the observed choice:}

\begin{itemize}
  \item By revealed preference, infer for all Brazilian own-account workers the lowest discount rate ($\rho$) that makes such occupation prefereable.
\end{itemize}

\end{frame} %%%%%%%%%%%%%%%%%%%%%%%%%%%%%%%%%%%%%%%%%%%%%%%%%%%%%%%%%%%%%%%%%%%%%

\begin{frame}%{Main result: 60\% of Brazilian OAWs are constrained}

\bigskip

\begin{columns}
  \column{0.44\textwidth} 
    \begin{figure}
      \caption{\hspace{.7cm} \textbf{CDF of discount rate lower bound} \\ \hspace{.7cm} (Brazil, urban areas, 2016-2019) \\ \hspace{.7cm} Source: Own estimation.}
      \includegraphics<1-3>[width=\textwidth]{../../exhibits/cdf_baseline_avg_ver_0.png}%
      \includegraphics<4>[width=\textwidth]{../../exhibits/cdf_baseline_avg_ver_1.png}%
      \includegraphics<5->[width=\textwidth]{../../exhibits/cdf_baseline_avg_ver_2.png}% 
    \end{figure}
    \bigskip 

  \column{0.02\textwidth}
  
	\column{0.55\textwidth}
	  \uncover<2->{
	  \textbf{When is the decision constrained?} \bigskip
	  \begin{itemize}
		  \item If the lowest rate compatible with the choice for OAW is above the market's.
	  \end{itemize} \bigskip}

	  \uncover<3->{
	  \textbf{Why?} \bigskip
	  \begin{itemize}
		  \item Pressing needs, restricted borrowing.
	  \end{itemize} \bigskip}

	  \uncover<6->{
	  \textbf{Main result:} \bigskip
	  \begin{itemize}
		  \item 60\% of OAWs in Brazil are constrained.
	  \end{itemize} \bigskip}

		\uncover<7->{
		\textbf{Implication:} \bigskip
		\begin{itemize}
		  \item Many rational workers can be \hl{stuck in low-pay OAW} given frictional job market, contrained finance, urgent consumption.
		\end{itemize}}

\end{columns}

\end{frame} %%%%%%%%%%%%%%%%%%%%%%%%%%%%%%%%%%%%%%%%%%%%%%%%%%%%%%%%%%%%%%%%%%%%%

\appendix

\begin{frame}{Appendix:}{Valuation equations in the extended job search model}

\begin{align*}
\text{\textcolor{main-color}{\fbox{\textbf{Wage employment}}}} & \quad \rho \cdot W(w) = w + \delta \cdot \Big( U - W(w) \Big) \phantom{\int} \\
& \\
\text{\textcolor{main-color}{\fbox{\textbf{Unemployment}}}} & \quad  \rho \cdot U = b + \lambda \cdot \int_{w_r}^{\infty} \Big( W(w) - U \Big) \, dF(w) \\
& \\
\text{\textcolor{main-color}{\fbox{\textbf{Reservation wage}}}} & \quad w_r = b + \frac{\lambda}{\delta + \rho} \cdot \int_{w_r}^{\infty} \Big( w - w_r \Big) \, dF(w) 
& \\
& \\
\text{\textcolor{main-color}{\fbox{\textbf{Own-account work}}}} & \quad \rho \cdot \mathit{OAW} = y 
\end{align*}

\end{frame} %%%%%%%%%%%%%%%%%%%%%%%%%%%%%%%%%%%%%%%%%%%%%%%%%%%%%%%%%%%%%%%%%%%%%

\begin{frame}{Appendix:}{The occupational decision from the perspective of the discount rate}

\begin{equation*}
\rho > \frac{\lambda}{y - b} \cdot \int_{w_r}^{\infty} \big( w - w_r \big) \, dF(w) - \delta
\end{equation*} 

\textbf{What is this?}
\begin{itemize}
\item The (right-hand side) \hl{minimum discount rate} that justifies the choice for OAW, given the individual productivity and the labor market conditions.
\end{itemize} 

\textbf{How is this useful?}

\begin{itemize}
\item Formalization of the idea that a sufficiently high urgency for consumption (i.e. the "necessity") can rationalize the choice for OAW \hl{for any value of $y$}.
\end{itemize}

\end{frame} %%%%%%%%%%%%%%%%%%%%%%%%%%%%%%%%%%%%%%%%%%%%%%%%%%%%%%%%%%%%%%%%%%%%%

\begin{frame}{Appendix:}{The building blocks of the structural model}

\begin{gather*}
\rho > \frac{\lambda}{y - b} \cdot \int_{w_r}^{\infty} \big( w - w_r \big) \, dF(w) - \delta \\
\downarrow \\
\hat{\rho}_i > \frac{\mathbb{E} \left( \lambda \,|\, X_i \right)}{y_i - \mathbb{E} \left( b \,|\, X_i \right)} \cdot \Big[ \mathbb{E}\left( w \,|\, w > w_r, X_i \right) - \mathbb{E} \left( w_r \,|\, X_i \right) \cdot \mathbb{P}(w \geq w_r) \Big] - \mathbb{E} \left( \delta \,|\, X_i \right)
\end{gather*}

\begin{enumerate}
\item \hl{$y_i$} is directly observable for own-account workers. 
\item \hl{$\mathbb{E} \left(\lambda \,|\, X_i \right)$} is fit with an unemp. duration model and with $\mathbb{P} (w \geq w_r)$. 
\item \hl{$\mathbb{E} \left(b \,|\, X_i \right)$} is assumed to be zero, the most frequent value. 
\item \hl{$\mathbb{E} \left( w \,|\, w > w_r, X_i \right)$} is fit with a Heckman selection model.  
\item \hl{$\mathbb{E} \left( w_r \,|\, X_i \right)$} is fit with a quantile regression at the 10th centile. 
\item \hl{$\mathbb{P} (w \geq w_r)$} is calculated for a normal distribution of wages. 
\item \hl{$\mathbb{E} \left(\delta \,|\, X_i \right)$} is fit with a job duration model.
\end{enumerate}

\end{frame} %%%%%%%%%%%%%%%%%%%%%%%%%%%%%%%%%%%%%%%%%%%%%%%%%%%%%%%%%%%%%%%%%%%%%

\begin{frame}{Appendix:}{The PNAD survey and the population of interest}

\begin{itemize}
\item \textbf{Data size:} 8.9 million observations (2.7 million individuals).
\item \textbf{Population of interest:} Adults living in urban areas.
\end{itemize} 

\begin{itemize}
\item \textit{Exclude individuals below 14 years old (\textasciitilde 19\% of population).}
\item \textit{Exclude individuals from rural areas (\textasciitilde 14\% of adults).}
\end{itemize} 

\begin{itemize}
\item \textbf{Sample size:} 5.3 million observations (1.6 million individuals).
\item \textbf{Time coverage:} 16 quarters (2016 Q1 to 2019 Q4).
\item \textbf{Monetary correction:} Inflation-adjusted values (\textasciitilde 4.2 p.p. yearly).
\item \textbf{Complex survey:} interview weights account for probability of observation.
\end{itemize} 

\end{frame} %%%%%%%%%%%%%%%%%%%%%%%%%%%%%%%%%%%%%%%%%%%%%%%%%%%%%%%%%%%%%%%%%%%%%

\end{document}

%%%%%%%%%%%%%%%%
%   End code   %
%%%%%%%%%%%%%%%%
